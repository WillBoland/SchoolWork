\documentclass{article}
\usepackage{amsthm}
\usepackage{amssymb}
\begin{document}
\title{Homework 5}
\author{Will Boland}
\maketitle

\textbf{Question 1}\newline
A)	2\newline
B)	4\newline
C)	1\newline
D)	1/2\newline
E)	-5\newline
F)	1, 2, 3\newline
G)	2\newline
H)	0, 1, 8\newline
I)	\{1\}\newline
J)	There are no members of the set $\varnothing$\newline
K)	\{1, 3\}\newline
L)	\{ 0, 1, 8\}\newline
M)	\{\{1\}, \{1, 2\}\}\newline
N) 	A $\supseteq$ C\newline
O) 	\{1, 2\}\newline
P)	\{0, 1, 8, 27 \}\newline\newline

\textbf{Question 2}\newline
A)	False\newline
B)	True\newline
C)	True\newline
D)	False\newline
E)	True\newline
F)	False\newline
G)	True\newline
H)	True\newline
I)	False\newline
J)	False\newline
K)	True\newline
L)	False\newline
M)	True\newline
N)	True\newline
O)	False\newline
P)	True\newline
Q)	False\newline
R)	False\newline
S)	True\newline
T)	False\newline
U)	False	\newline\newline

\textbf{Question 3}\newline
A)	Yes, this statement is true because 5 $\mathbb{\in}$ A and 6 $\mathbb{\in}$ B, which combined, as P required, equal 11.\newline
B)	No, this statement is false because  5 $\mathbb{\in}$ A, which is the largest value in A, and 8 $\mathbb{\in}$ B (largest value in b), which combined, equal 13, one less than 14 as required.\newline
C)	False. 1 $\mathbb{\in}$ D; however, 1 $\mathbb{\notin}$ S\newline
D)	True. The order of the members of the set does not matter, and all of the members are the same in C and as defined as \{3, 5, 1\}\newline
E)	True. As stated in the above letter, the order of the members of the set does not matter, but the amount of times a member appears in the set also does not matter.\newline
F)	False. $\mathbb{\varnothing}$ = \{\}, so $\mathbb{\varnothing}$ $\neq$ \{$\mathbb{\varnothing}$\}\newline
G)	False because ) is a member of B but not A.\newline
H)	True because the members of C (1, 3, 5) all are members of X\newline
I)	False because 5 is a member of C, but not a member of Y\newline
J)	True because all the members of the empty set appear in B\newline
K)	True because \{1\} and \{2, 1\} appear in S\newline
L)	False because \{1, 2\} is a member in S\newline
M)	False because 1 does not appear in the powerset of S\newline
N)	false because \{1\} does not in the powerset of S\newline
O)	True because \{\{1\}\} does appear in the powerset of S\newline
P)	False because \{\{\{1\}\}\} does not appear in the powerset of S\newline\newline

\textbf{Question 4}\newline
A)	$\overline{\rm B}$ = \{1, 3, 5, 7, 9, 10\}\newline
B)	$\overline{\rm C\cup D}$$\cap$\{2, 3, 4\} = \{4\}\newline
C)	$\overline{\rm \varnothing}$ \ \{0, 1, 2, 3, 4, 5, 6, 7, 8, 9, 10\}\newline
D)	$\overline{\rm \{0, 1, 2, 3, 4 , 5, 6, 7, 8, 9 ,10\}}$ = \{\}\newline
E)	(A $\setminus$B) $\cap$ D = \{1, 3\}\newline
F)	A $\cap$ C = \{1, 3, 5\}\newline
G)	A $\cup$ C = \{1, 2, 3, 4, 5\}\newline
H)	A $\setminus$ C = \{2, 4\}\newline
I)	C $\setminus$ A = None\newline
J)	B $\cup$ C = \{\}\newline
K)	B $\setminus$ C = B\newline
L)	$\wp$(C) = \{\{1\}, \{3\}, \{5\}, \{1, 3\}, \{1, 5\}, \{3, 5\}, \{1, 3, 5\}, \{\}\}\newline\newline

\textbf{Question 5}\newline
B = \{$x$ $\mid$ x $\in$ $\mathbb{N}$ $\wedge$ $x$ is even $\wedge$ $x$ $\leq$ 8\}\newline\newline

\textbf{Question 6 (Bonus)}\newline
B = \{$x$ $\mid$ x $\in$ $\mathbb{N}$ $\wedge$ $x$ is not odd $\wedge$ $x$ $\leq$ 8\}\newline\newline

\textbf{Question 7}\newline
A)	$\mid$B$\mid$ = 5\newline
B)	$\mid$S$\mid$ = 4\newline
C)	$\mid$\{$x$ $\mid$ $x$ $\in$ $\mathbb{N}$ $\wedge$ $x$ $\leq$ 4\}$\mid$ = 5\newline
D)	$\mid$\{$x$ $\mid$ $x$ $\in$ $\mathbb{N}$ $\wedge$ $x$ $\leq$ 1000\}$\mid$ = 1001\newline
E)	$\mid$$\emptyset$$\mid$ = 0\newline
F)	$\mid$Q$\mid$ = Infinite\newline
G)	$\mid$$\wp$(C)$\mid$ = 8\newline
H)	$\mid$$\wp$(A)$\mid$ = 32\newline
I)	$\mid$$\wp$(Q)$\mid$ = Infinite\newline


\enddocument
















