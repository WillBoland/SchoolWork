\documentclass{article}
\usepackage{amsthm}
\usepackage{amssymb}
\begin{document}
\title{Homework 8}
\author{Will Boland}
\maketitle

\textbf{Question 1}\newline
$H$$\rightarrow$($T$$\wedge$$\neg$$R$)\newline\newline

\textbf{Question 2}\newline
A)	$\exists$($P$(x)$\wedge$$\neg$$E$(x))\newline
B)	All positive numbers are even.\newline
C)	(1,3,5,9)\newline\newline

\textbf{Question 3}\newline
A)	fml2 = A$\wedge$B, fml1 = A $\lor$ B\newline
B)	\{2/3, 4/3, -2/3\} $\subseteq$ A$\setminus$$\mathbb{Z}$\newline
C)	\{-2\} $\in$ $P$(B)\newline\newline

\textbf{Question 4}\newline
\textbf{Claim: } (R$\wedge$S)$\rightarrow$$\neg$Q, R$\rightarrow$S $\vdash$ (R$\wedge$P)$\rightarrow$$\neg$(P$\wedge$Q)\newline

\textbf{\textit{Proof}}\newline
Assume (R$\wedge$S)$\rightarrow$$\neg$Q and R$\rightarrow$S\newline
|	Assume R$\wedge$P, and Q (to show a contradiction)\newline
|	From R$\wedge$P, R and P. ($\wedge$-elimination)\newline
|	From P and Q, P$\wedge$Q ($\wedge$-introduction)\newline
|	Due to R$\rightarrow$S and R, S ($\rightarrow$-elimination)\newline
|	From R and S, R$\wedge$S ($\wedge$-introduction)\newline
|	Due to (R$\wedge$S)$\rightarrow$$\neg$Q and R$\wedge$S, $\neg$Q ($\rightarrow$-elimination)\newline
Assuming R$\wedge$P and Q, we got both Q and $\neg$Q, as well as P$\wedge$Q, which contradict, so $\neg$(P$\wedge$Q); therefore (R$\wedge$P)$\rightarrow$$\neg$(P$\wedge$Q) (proof by contradiction/direct proof)$\square$\newline\newline

\textbf{Question 5}\newline
A)	False. (A=T, C=F, B=F, D=T)\newline\newline

B)	\textbf{Claim: }(A$\wedge$B)$\rightarrow$(C$\lor$D) $\equiv$ (A$\rightarrow$C)$\lor$(B$\rightarrow$D)\newline

\textbf{\textit{Proof}}\newline
(A$\wedge$B)$\rightarrow$(C$\lor$D) $\equiv$ $\neg$(A$\wedge$B) $\lor$ (C$\lor$D) (Material implication)\newline
$\equiv$ $\neg$A $\lor$ $\neg$B $\lor$ (C$\lor$D) (Demorgans)\newline
$\equiv$ $\neg$A $\lor$ C $\lor$ $\neg$B$\lor$D ($\lor$ associative)\newline
$\equiv$ (A$\rightarrow$C) $\lor$ ($\neg$B$\lor$D) (Implication)\newline
$\equiv$ (A$\rightarrow$C) $\lor$ (B$\rightarrow$D) (Implication)$\square$\newline\newline

\textbf{Question 6}\newline
\textbf{Claim: } For all sets A, B, C, and D, if A$\cup$C$\subseteq$D, then A$\setminus$B$\subseteq$D$\setminus$B\newline
\textbf{\textit{Proof: }}\newline
Choose sets A, B, C, and D, and assume  A$\cup$C$\subseteq$D
|	Choose $x$$\in$A$\setminus$B and A$\cup$C.\newline
|	So $x$$\in$A and $x$$\notin$B.\newline
|	From A$\cup$C, $x$$\in$A or $x$$\in$D\newline
||		Case 1: Suppose $x$$\in$A\newline
||		From this, we can directly prove $x$$\in$D$\cap$A\newline
||		Case 2: Suppose $x$$\in$C\newline
||		From this, we can directly prove $x$$\in$D$\cap$A\newline
|		Either case, we proved  $x$$\in$D$\cap$A\newline
Therefore A$\setminus$B$\subseteq$D$\setminus$B $\square$


\end{document}