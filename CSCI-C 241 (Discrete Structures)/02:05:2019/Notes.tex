\documentclass{article}
\usepackage{amsthm}
\usepackage{amsmath}
\usepackage{amssymb}
\begin{document}
\title{Notes}
\author{Will Boland}
\maketitle

\textbf{Set Theory}\newline
A \underline{set} is defined by which things are its \underline{members} (or elements).\newline
To say that x is a member of the set X, we write x $\in$ X,\newline\newline

\textbf{Ways to define sets}\newline
English \newline
"A is the set of all integers between 0 and 3 (inclusive)"\newline\newline

\underline{Setlist Notation}\newline
A = \{0,1,2,3\}\newline
B = \{0,1,2,3,...,99,100\}\newline\newline

P = \{3,4,6,8,12,14,18,20,24,30,38,...\}\newline
= the set of all \#'s that are one more han a prime \#\newline
= \{p + 1 $\mid$ p is a prime \#\}\newline\newline

\underline{Set builder notation}\newline
S = \{$n^2$ $\mid$ n is an integer $\wedge$ n $\geq$ 20 $\wedge$ n $\leq$ 500\}\newline\newline

\textbf{\underline{Special Sets of Numbers}}\newline
\underline{Natural Numbers}\newline
$\mathbb{N}$ = \{0,1,2,3,...\}\newline
\underline{Integers}\newline
$\mathbb{Z}$ = \{...,-2,-1,0,1,2,...\}\newline
\underline{Rational Numbers}\newline
$\mathbb{Q}$ = \{$\dfrac{p}{q}$ $\mid$ p $\in$ $\mathbb{Z}$ $\wedge$ q $\in$ $\mathbb{Z}$ $\wedge$ q $\neq$ 0\}\newline
= any \# that can be written as a ratio/fraction/quotiant of integers\newline\newline

\underline{The empty set}\newline
X = \{\}\newline
X = the set with no numbers\newline
The symbol for this: $\emptyset$\newline\newline


\textbf{Set Operations}\newline
A$\cup$B = \{X $\mid$ X$\in$A $\lor$ x$\in$B\} This is called a union.\newline
A$\cap$B = \{x $\mid$ x$\in$A $\wedge$ x$\in$B\} This is called the intersection. \newline

 = \{x $\mid$ x$\in$A\} This is called the complement. (depends on context, or the universal set) \newline\newline

The universal set (or the universe) is the set of all the things we currently care about. We often use a cursive capital $\mathbb{U}$ for the universe.



\enddocument