\documentclass{article}
\usepackage{amsthm}
\usepackage{amssymb}
\usepackage{tabto}
\begin{document}
\title{Notes}
\author{Will Boland}
\maketitle

\textbf{Claim: }(A$\wedge$B)$\rightarrow$C, A $\vdash$ (B$\wedge$D)$\rightarrow$C is valid)\newline

\textbf{\textit{Proof:}}\newline
Suppose (A$\wedge$B)$\rightarrow$C and A.\newline
|	Assume B$\wedge$D\newline
|	From B$\wedge$D, we get B and D ($\wedge$-elim.)\newline
|	A and B together imply A$\wedge$B ($\wedge$-intro)\newline
|	We have (A$\wedge$B)$\rightarrow$C and A$\wedge$B, so C ($\rightarrow$-elim)\newline
Assuming B$\wedge$D, we proved C and therefore (B$\wedge$D)$\rightarrow$C. (direct proof or $\rightarrow$-introduction)\newline\newline

\textbf{Claim: } (X$\lor$Y)$\wedge$$\neg$$\neg$Z $\vdash $X$\rightarrow$Z\newline

\textbf{\textit{Proof:}}\newline
Assume (X$\lor$Y)$\rightarrow$$\neg$$\neg$Z\newline
|	Assume X$\lor$Y\newline
|	Apply (X$\lor$Y)$\rightarrow$$\neg$$\neg$Z, to X$\lor$Y, resulting in $\neg$$\neg$Z. (Appl.)\newline
|	$\neg$$\neg$Z leads to Z (double neg)\newline
Under the assumption X$\lor$Y, we orived Z, and si (X$\lor$Y)$\rightarrow$Z (Dir. proof)\newline\newline

\textbf{Do this instead of above!!!}:\newline\newline

Assume (X$\lor$Y)$\rightarrow$$\neg$$\neg$Z\newline
|	Assume X\newline
|	From X, we know X$\lor$Y (weak.)\newline
|	(X$\lor$Y)$\rightarrow$$\neg$$\neg$Z and X$\lor$Y imply $\neg$$\neg$Z (appl.)\newline
|	Because $\neg$$\neg$Z, Z. (double neg)\newline
We used X to prove Z, and hence X$\rightarrow$Z (Dir. pf)$\square$

\textbf{When you pick your assumption, think about what you will do after you have finished the subproof, \underline{not} what you will do \underline{inside} the subproof.}\newline
\end{document}