\documentclass{article}
\begin{document}
\title{Homework}
\author{Will Boland}
\maketitle

A \textbf{direct proof } is a sequence of deductions starting by assuming the premises and eventually proving the conclusion.\newline\newline

\textbf{Semi-formal proofs}\newline
- Limited set of rules\newline
- One rule per step\newline
- Always say what formulas you are using\newline
- Use english sentences\newline
- clearly write down the assumptions\newline
- cite the names of the logical rules \newline\newline

\textbf{Claim: this is valid}\newline
$\neg$$\neg$A\newline
A$\rightarrow$(B$\wedge$C)\newline
-------------------\newline
A$\wedge$C\newline\newline

This can also be read as $\neg$$\neg$A, A$\rightarrow$(B$\wedge$C) $\vdash$ A$\wedge$C\newline\newline

The turnstile symbol $\vdash$ means there is a proof with assumptions p1,...pn, and conclusion c.\newline\newline

\textbf{Pf:}\newline
Assume $\neg$$\neg$A and A$\rightarrow$(B$\wedge$C) are true.\newline
Since $\neg$$\neg$A is true, so is A. (Double Negation)\newline
Because A$\rightarrow$(B$\wedge$C) and A, we know B$\wedge$C. (Appl.)\newline
From B$\wedge$C, we can conclude C. ($\wedge$-Elim.)\newline
A and C together imply A$\wedge$C. ($\wedge$-Elim.) End the proof with a box like a period.\newline\newline

\textbf{Natural Deduction Rules}\newline
All of the Natural Deduction rules are inference rules, meaning that they only work in one direction, and they only work aon entire formulas, not parts of forumulas.\newline\newline
$\wedge$-Elimination or simplification. \newline
(If you know that p$\wedge$q is true, then you can conclude p is true, or q is true, or both.) \newline
p$\wedge$q $\vdash$ p\newline
P$\wedge$q $\vdash$ q\newline\newline

$\wedge$-Introduction or conjuction. \newline
p,q $\vdash$ q$\wedge$p\newline\newline

$\rightarrow$-Elimination or Applicaiton or Modus Ponanus. \newline
p$\rightarrow$q by itself is useless. p$\rightarrow$q together with p, implies q.\newline
p$\rightarrow$q, p$\vdash$q\newline
This DOES NOT mean that you can conclude that B == (A$\wedge$B)$\rightarrow$C due to $\rightarrow$-Elimination!!!\newline\newline

$\neg$-Elimination or Double Negation.\newline
$\neg$$\neg$p $\vdash$ p\newline\newline

Weakening or $\lor$-Introduction.\newline
p$\vdash$p$\lor$q\newline
p$\vdash$q$\lor$p\newline\newline

\textbf{Claim: (A$\lor$B)$\rightarrow$C, A $\vdash$ C)}\newline
\textbf{(Pf:)}\newline
Assume (A$\lor$B)$\rightarrow$C and A. (Weak)\newline
Apply (A$\lor$B)$\rightarrow$ C to A$\lor$ B to get C (Appl.)\newline\newline

Since X, we can conclude $\neg$(Z$\rightarrow$$\neg$W)$\lor$X. (Weak). \newline\newline

\end{document}